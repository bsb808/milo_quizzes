\documentclass[12pt,onecolumn]{article}
%\documentclass[11pt,draftcls,journal,onecolumn]{./latex_ieee/IEEEtran}

\special{papersize=8.5in,11.0in}
\usepackage{fullpage}
\usepackage{subfigure}
\usepackage[pdftex]{graphicx}
\usepackage{cite}      % Written by Donald Arseneau
\usepackage{graphicx}  % Written by David Carlisle and Sebastian Rahtz
\usepackage{url}       % Written by Donald Arseneau
\usepackage{stfloats}  % Written by Sigitas Tolusis
\usepackage{amssymb,longtable,dcolumn}


\newenvironment{menumerate}{\begin{enumerate}\addtolength{\itemsep}{1.0em}}{\end{enumerate}}

\usepackage[left=1in, right=2.5in, top=1in]{geometry}

\newcommand{\fiblank}{$\rule{1.5cm}{0.15mm}\:$}

\begin{document}
\graphicspath{{images/}}

\centerline{{\Large {\bf Math}}}


\begin{menumerate}
\item For each number, classify in as many categories (natural, whole, integer, rational, irrational, real) as possible
  \begin{menumerate}
  \item 3
  \item -4.5
  \item 3.1415917452301\ldots
  \end{menumerate}
  \item $(-3)(-4)=$
  \item $50 \times -13 = $
  \item $ 44  \div -4 = $
  \item Write the ratio as a fration: For every 50 video games, 2 are blockbusters.
  \item Solve for the unknown: $ \frac{2}{20} = \frac{x}{100}$
    \item Solve for the unknown: $\frac{n}{2.5}=\frac{10}{5}$
  \item 0.5 feet = \fiblank inches
  \item 2 miles = \fiblank kilometers
  \item Write 15\% as a fraction.
  \item Write 85\% as a decimal.
  \item $\frac{5}{20}$ is what percent?
  \item What is 10\% of 200?
  \item 35 is 20\% of what number?  
  \end{menumerate}

\end{document}
